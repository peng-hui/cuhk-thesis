\renewcommand{\ttdefault}{pxtt}

\newcommand{\URL}{\url}
\newcommand{\cc}[1]{{\smaller[0.5]\texttt{#1}}}

% enable the below for ACM camera ready
%\clubpenalty=10000
%\widowpenalty=10000
%\renewcommand*{\bibfont}{\raggedright}

%\linespread{1.2}

\fvset{fontsize=\scriptsize,xleftmargin=8pt,numbers=left,numbersep=5pt}

\input{code/fmt}
\newcommand{\figrule}{\hrule width \hsize height .33pt}
\newcommand{\coderule}{\vspace{-0.5em}\figrule\vspace{0.2em}}

\setlength{\abovedisplayskip}{0pt}
\setlength{\abovedisplayshortskip}{0pt}
\setlength{\belowdisplayskip}{0pt}
\setlength{\belowdisplayshortskip}{0pt}
\setlength{\jot}{0pt}

\def\Snospace~{\S{}}
\renewcommand*\sectionautorefname{\Snospace}
\def\sectionautorefname{\Snospace}
\def\subsectionautorefname{\Snospace}
\def\subsubsectionautorefname{\Snospace}
\def\chapterautorefname{\Snospace}
%\renewcommand{\figurename}{Fig.}
%\def\figureautorefname{\figurename}
\newcommand{\subfigureautorefname}{\figureautorefname}

%\numberwithin{equation}{section}
\newcommand{\yes}{Y}
\newcommand{\no}{}

% sema
\newcommand{\shl}{\ \cc{<}\cc{<}\ }
\newcommand{\shr}{\ \cc{>}\cc{>}\ }
\newcommand{\x}{$\times$\xspace}

\if 0
\renewcommand{\topfraction}{0.9}
\renewcommand{\dbltopfraction}{0.9}
\renewcommand{\bottomfraction}{0.8}
\renewcommand{\textfraction}{0.05}
\renewcommand{\floatpagefraction}{0.9}
\renewcommand{\dblfloatpagefraction}{0.9}
\setcounter{topnumber}{10}
\setcounter{bottomnumber}{10}
\setcounter{totalnumber}{10}
\setcounter{dbltopnumber}{10}
\fi

\newif\ifdraft\drafttrue
\newif\ifnotes\notestrue
\ifdraft\else\notesfalse\fi

\newcommand{\ie}{{\em i.e.}}
\newcommand{\eg}{{\em e.g.}}
\newcommand{\aka}{{\em a.k.a}}
\newcommand{\etal}{{\em et al.}}
\newcommand{\etc}{{\em etc.}}

% hide comments
% \renewcommand{\TK}[1]{\ignorespaces}
% \renewcommand{\XXX}[1]{\ignorespaces}
% \renewcommand{\TODO}[1]{\ignorespaces}

%% Ensure ligatures (e.g., ``fine official flag'') can be copy/pasted from PDF.
%\input{glyphtounicode}
%\pdfgentounicode=1

\newcolumntype{R}[1]{>{\raggedleft\let\newline\\\arraybackslash\hspace{0pt}}p{#1}}

% include macros
\newcommand{\includepdf}[1]{
  \includegraphics[width=\columnwidth]{#1}
}
\newcommand{\includeplot}[1]{
  \resizebox{\columnwidth}{!}{\input{#1}}
}

% list
\newcommand{\squishlist}{
\begin{itemize}[noitemsep,nolistsep] %,leftmargin=10pt]
  \setlength{\itemsep}{-0pt}
}
\newcommand{\squishend}{
  \end{itemize}
}

%%
%% NOTE.
%%  to use circled number in caption, use
%%   (e.g., \protect\WC{1})
%%
\usepackage{tikz}
\newcommand*\WC[1]{%
\begin{tikzpicture}[baseline=(C.base)]
\node[draw,circle,inner sep=0.2pt](C) {#1};
\end{tikzpicture}}

\newcommand*\BC[1]{%
\begin{tikzpicture}[baseline=(C.base)]
\node[draw,circle,fill=black,inner sep=0.2pt](C) {\textcolor{white}{#1}};
\end{tikzpicture}}

\usepackage{xstring}
\newcommand{\PP}[1]{
\vspace{2bp}
\noindent{{\bf {\IfEndWith{#1}{.}{#1}{#1.}}}}
}
\newcommand{\subsubsubsection}[1]{
\vspace{2bp}
\noindent{\underline{#1}\hfill\break}
}

\newcommand{\PN}[1]{
\vspace{2bp}
\noindent{\bf #1}
}

\newcommand{\ra}[1]{\renewcommand{\arraystretch}{#1}}
\newcommand{\V}{\checkmark}
\newcommand{\X}{{\footnotesize $\times$}\xspace}
\renewcommand{\O}{\phantom{0}}

%% units
\newcommand{\B}{\,\text{B}\xspace}
\newcommand{\K}{\,\text{K}\xspace}
\newcommand{\M}{\,\text{M}\xspace}
\newcommand{\T}{\,\text{T}\xspace}
\newcommand{\KB}{\,\text{KB}\xspace}
\newcommand{\MB}{\,\text{MB}\xspace}
\newcommand{\GB}{\,\text{GB}\xspace}
\newcommand{\TB}{\,\text{TB}\xspace}

\newcommand{\Bs}{\,\text{B/s}\xspace}
\newcommand{\KBs}{\,\text{KB/s}\xspace}
\newcommand{\MBs}{\,\text{MB/s}\xspace}
\newcommand{\GBs}{\,\text{GB/s}\xspace}

% boxbeg/end
\newcommand{\boxbeg}{
\vspace{2bp}
\noindent\begin{tabular}{|l|}\hline
\begin{minipage}{0.98\textwidth}
\vspace{2bp}
\noindent
}

\newcommand{\boxend}{
\vspace{2bp}
\end{minipage}\\ \hline
\end{tabular}
\vspace{-10pt}
}

\newcommand{\blstinline}[1]{\lstinline[numbers=none, caption={}, frame={},stringstyle=\color{black}, deletekeywords={md5, foreach,strcmp, array, stripslashes},identifierstyle=\color{black},commentstyle=\color{black}]{#1}} 

\def\degree#1{\def\degree{#1}}
\def\degreezh#1{\def\degreezh{#1}}
\def\programme#1{\def\programme{#1}}
\def\programmezh#1{\def\programmezh{#1}}
\def\supervisor#1{\def\supervisor{#1}}
\def\thesistitle#1{\def\thesistitle{#1}}
\def\thesistitlezh#1{\def\thesistitlezh{#1}}
\def\authorname#1{\def\authorname{#1}}
\def\authornamezh#1{\def\authornamezh{#1}}
\def\submitdate#1{\def\submitdate{#1}}
\def\institution#1{\def\institution{#1}}
\def\institutionzh#1{\def\institutionzh{#1}}
\def\committee#1{\def\committee{#1}}
\def\abstract#1{\def\abstract{#1}}
\def\abstractzh#1{\def\abstractzh{#1}}
\def\acknowledgement#1{\def\acknowledgement{#1}}
\def\dedication#1{\def\dedication{#1}}

\def\coverpage{
  % no page number for the cover page
  \thispagestyle{empty}

  \vspace*{2cm}
  \begin{center}
    \LARGE {\bf \thesistitle}

    \vskip 1cm
    \large \textbf{\authorname}

    \vskip 1cm
    \large
  % both Fulfilment and Fulfillment are correct
    A Thesis Submitted in Partial Fulfilment \\
    of the Requirements for the Degree of \\
    \degree \\
    in \\
    \programme \\

    %\vfill
    \vskip 1.5cm
    \large \institution \\
    \submitdate
  \end{center}

  \newpage

  \thispagestyle{empty}
  \vspace*{10mm}
  \begin{center}
    \large
    \underline{\bf Thesis Assessment Committee}
    \vskip 1.5cm 
    \committee
    \vfill
  \end{center}
  \normalsize
  \newpage
}

%=======================================================

\def\abstractpage{
  % XXX a known bug: click abstract in menu will not 
  % direct to correct position
  \addcontentsline{toc}{chapter}{Abstract}
  \noindent Abstract of thesis entitled: \\
  \expandafter\ul\expandafter{\thesistitle} \\
  Submitted by \underline{\authorname} \\
  for the degree of \underline{\degree} \\
  at \institution~in \submitdate
  \vskip 0.5cm \noindent
  \input{\abstract}
  \vfill

  \newpage

  %\vspace*{1.5cm}
  %\chapter*{摘要}
  %\addcontentsline{toc}{chapter}{摘要}
  \noindent 論文題目:\underline{\thesistitlezh}\\
  \noindent 作者:\underline{\authornamezh} \\
  \noindent 學校:\underline{\institutionzh} \\
  \noindent 學系:\underline{\programmezh} \\
  \noindent 修讀學位:\underline{\degreezh} \\
  \noindent 摘要:

  \vskip 0.5cm
  \input{\abstractzh}
  \clearpage
}

%=======================================================

\def\acknowledgementpage{%
  \chapter*{Acknowledgement}
  \addcontentsline{toc}{chapter}{Acknowledgement}
  \input{\acknowledgement}
  \newpage
}

%=======================================================

\def\dedicationpage{%
  \vspace*{1cm}
  \vfill
  \begin{center}
      \input{\dedication}
  \end{center}
  \vfill
  \newpage
}


%=======================================================

\newcommand{\snote}[2]{%
  \vspace*{3mm}
  \begin{center}
    \framebox[13.6cm][t]{%
      \hspace*{2mm}
      \parbox[t]{12.8cm}{%
        \vspace*{2mm}
        \centerline{\bf #1}\vspace*{-2mm}
        \rule{12.8cm}{0.4mm} 
        \\#2
        \vspace*{3mm}
      }
    }
  \end{center}
  \vspace*{3mm}
}

\newcommand{\chapterend}{
  \vfill
  \noindent\rule{\textwidth}{0.5mm} \\
  \noindent{\large\bf $\Box$ End of chapter.}
}

\newcommand{\reading}[1]{
  \vfill
  \noindent {\bf $\equiv$ Readings:}  #1
}

\newcommand*\emptycirc[1][1ex]{\tikz\draw (0,0) circle (#1);} 
\newcommand*\halfcirc[1][1ex]{%
  \begin{tikzpicture}
  \draw[fill] (0,0)-- (90:#1) arc (90:270:#1) -- cycle ;
  \draw (0,0) circle (#1);
  \end{tikzpicture}}
\newcommand*\fullcirc[1][1.05ex]{\tikz\fill (0,0) circle (#1);} 
\counterwithout{footnote}{chapter}
