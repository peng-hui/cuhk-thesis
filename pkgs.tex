\usepackage[hyphens]{url}
\urlstyle{same}
\usepackage[breaklinks,colorlinks]{hyperref}
\usepackage[table,usenames,dvipsnames]{xcolor}
\hypersetup{citecolor=Green,linkcolor=Blue, urlcolor=Blue}
\let\Bbbk\relax
\usepackage{amsmath,amsopn,amssymb}
\usepackage{subfigure}
\usepackage{endnotes,microtype,xspace,graphicx,fancyvrb,multirow}
\usepackage{booktabs}
\usepackage{array,underscore,relsize}
\usepackage[T1]{fontenc}
%\usepackage{times}
%\usepackage{mathptmx}
\usepackage{fancyhdr}
\usepackage{enumitem}
\usepackage[labelfont=bf,font=small,skip=5pt]{caption}
%\pagestyle{fancy}
\fancyhf{}
\renewcommand{\headrulewidth}{0pt}
%\cfoot{\thepage}

% for math macro and numbers
\usepackage{fp}
\usepackage{siunitx}

% balance bibliography
\usepackage{balance}

% use \num{123456} -> 123,456
\sisetup{group-separator={,},group-minimum-digits={3},output-decimal-marker={.}}

% comment environment
\usepackage{esvect}
\usepackage{verbatim}

\usepackage{listings}
\usepackage{color}
% algorithm
\usepackage[ruled, vlined, inoutnumbered, linesnumbered]{algorithm2e}

\usepackage[]{minted} %% no need of the outputdir for VScode/Overleaf user
\usemintedstyle{default}
\providecommand*{\listingautorefname}{Listing} %% define an autoref 
% make minted listing and lstlisting share the same counter
\counterwithin{listing}{chapter}
\AtBeginEnvironment{listing}{\setcounter{listing}{\value{lstlisting}}} 
\AtEndEnvironment{listing}{\stepcounter{lstlisting}}

\def\degree#1{\def\degree{#1}}
\def\degreezh#1{\def\degreezh{#1}}
\def\programme#1{\def\programme{#1}}
\def\programmezh#1{\def\programmezh{#1}}
\def\supervisor#1{\def\supervisor{#1}}
\def\thesistitle#1{\def\thesistitle{#1}}
\def\thesistitlezh#1{\def\thesistitlezh{#1}}
\def\authorname#1{\def\authorname{#1}}
\def\authornamezh#1{\def\authornamezh{#1}}
\def\submitdate#1{\def\submitdate{#1}}
\def\institution#1{\def\institution{#1}}
\def\institutionzh#1{\def\institutionzh{#1}}
\def\committee#1{\def\committee{#1}}
\def\abstract#1{\def\abstract{#1}}
\def\abstractzh#1{\def\abstractzh{#1}}
\def\acknowledgement#1{\def\acknowledgement{#1}}
\def\dedication#1{\def\dedication{#1}}

\def\coverpage{
  % no page number for the cover page
  \thispagestyle{empty}

  \vspace*{2cm}
  \begin{center}
    \LARGE {\bf \thesistitle}

    \vskip 1cm
    \large \textbf{\authorname}

    \vskip 1cm
    \large
  % both Fulfilment and Fulfillment are correct
    A Thesis Submitted in Partial Fulfilment \\
    of the Requirements for the Degree of \\
    \degree \\
    in \\
    \programme \\

    %\vfill
    \vskip 1.5cm
    \large \institution \\
    \submitdate
  \end{center}

  \newpage

  \thispagestyle{empty}
  \vspace*{10mm}
  \begin{center}
    \large
    \underline{\bf Thesis Assessment Committee}
    \vskip 1.5cm 
    \committee
    \vfill
  \end{center}
  \normalsize
  \newpage
}

%=======================================================

\def\abstractpage{
  %\chapter*{Abstract}
  \addcontentsline{toc}{chapter}{Abstract}
  \noindent Abstract of thesis entitled: \\
  \underline{\thesistitle} \\
  Submitted by \underline{\authorname} \\
  for the degree of \underline{\degree} \\
  at \institution~in \submitdate
  \vskip 0.5cm \noindent
  Enghlish abstract

  \vfill

  \newpage

  %\vspace*{1.5cm}
  %\chapter*{摘要}
  %\addcontentsline{toc}{chapter}{摘要}
  \noindent 論文題目:\underline{\thesistitlezh}\\
  \noindent 作者:\underline{\authornamezh} \\
  \noindent 學校:\underline{\institutionzh} \\
  \noindent 學系:\underline{\programmezh} \\
  \noindent 修讀學位:\underline{\degreezh} \\
  \noindent 摘要:

  \vskip 0.5cm
  \input{\abstractzh}
  \clearpage
}

%=======================================================

\def\acknowledgementpage{%
  \chapter*{Acknowledgement}
  \addcontentsline{toc}{chapter}{Acknowledgement}
  \input{\acknowledgement}
  \newpage
}

%=======================================================

\def\dedicationpage{%
  \vspace*{1cm}
  \vfill
  \begin{center}
    Dedication to XXX
  \end{center}
  \vfill
  \newpage
}

%=======================================================

\newcommand{\snote}[2]{%
  \vspace*{3mm}
  \begin{center}
    \framebox[13.6cm][t]{%
      \hspace*{2mm}
      \parbox[t]{12.8cm}{%
        \vspace*{2mm}
        \centerline{\bf #1}\vspace*{-2mm}
        \rule{12.8cm}{0.4mm} 
        \\#2
        \vspace*{3mm}
      }
    }
  \end{center}
  \vspace*{3mm}
}

\newcommand{\chapterend}{
  \vfill
  \noindent\rule{\textwidth}{0.5mm} \\
  \noindent{\large\bf $\Box$ End of chapter.}
}

\newcommand{\reading}[1]{
  \vfill
  \noindent {\bf $\equiv$ Readings:}  #1
}
